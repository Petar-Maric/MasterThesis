\documentclass[11pt,a4paper,english]{article} % document type and language

\usepackage[main=ngerman, english]{babel}   % multi-language support (Hauptsprache GER)
\babeltags{en = english}
\usepackage[utf8]{inputenc} % wegen deutschen Umlauten
\usepackage{float}   % floats
\usepackage{url}     % urls
\usepackage{hyperref} % klickbare links
\usepackage[style=authoryear, sorting=nyt]{biblatex} % getting the biblatex usepackage
\addbibresource{bibliography.bib} % locates the bib file


\title{Master Thesis - Preliminary } %Titel des Dokument
\author{Petar Maric}% Autor des Dokument
\date{\today} %Datum

% Zitatbeispiel
% Hallo Welt gäääähn \parencite[vgl.][S. 100]{Reich2013}
%-----------------------------------------------------------------------------------------------
\begin{document}
\maketitle
\newpage
%----------------------------------------------------------------------------------------------
\tableofcontents
\newpage
%----------------------------------------------------------------------------------------------
\section{Einführung}
% Notizen zur Einführung
\begin{comment}
- Thema und Aufgabenstellung
  ○ Thema
    § Aktuelle bzw. Historische Bezüge herstellen
    § Größeren Bezugsrahmen darstellen
      □ Vom allg. Meinen auf meine Aufgabenstellung kommen
  ○ Aufgabenstellung
    § Wichtigste Punkte aus der offiziellen Aufgabenstellung mit der eigenen Formulierung umschreiben (ggf. Offizielle AGST in die Arbeit einfügen)
    § Ziele der Untersuchung mit themenrelevanten Teilfragen
    § Abgrenzung des Untersuchungsgegenstandes mit Begründung
- Vorgehensweise
  ○ Gang der Untersuchung darlegen
    § Welche Schritte bin ich gegangen um an mein Ziel zu kommen
  ○ Vorgehensweise darlegen (Orientierung an der Gliederung)
  ○ Begründung zur Vorgehensweise liefern
    § Input warum ich so und nicht anders vorgegangen bin
- Ausschluss/ Eingrenzung
  ○ Beschreiben welche Themengebiete nich behandelt werden
    § Begründen
  ○ Relevanz des Themas unterstreichen
  ○ Nach Fertigstellung der Arbeit
    § Erneut prüfen

\end{comment}

\section{Stand der Technik}

%---------------------------

% document contents



\newpage
\printbibliography
\end{document}
