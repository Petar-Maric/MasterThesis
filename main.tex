\documentclass[11pt,a4paper,english]{article} % document type and language

\usepackage[main=ngerman, english]{babel}   % multi-language support (Hauptsprache GER)
\babeltags{en = english}
\usepackage[utf8]{inputenc} % wegen deutschen Umlauten
\usepackage{float}   % floats
\usepackage{url}     % urls
\usepackage{hyperref} % klickbare links
\usepackage[style=authoryear, sorting=nyt]{biblatex} % getting the biblatex usepackage
\addbibresource{bibliography.bib} % locates the bib file
\usepackage{verbatim} % Block Kommentare mit \beign{comment} .... \end{comment}
\usepackage{geometry}
\geometry{
  left=2.5cm,
  right=2cm,
  top=2.5cm,
  bottom=2cm,
  bindingoffset=5mm
}



\title{Master Thesis - Preliminary } %Titel des Dokument
\author{Petar Maric}% Autor des Dokument
\date{\today} %Datum

% Zitatbeispiel
% Hallo Welt gäääähn \parencite[vgl.][S. 100]{Reich2013}
%-----------------------------------------------------------------------------------------------
\begin{document}
\maketitle
\newpage
%----------------------------------------------------------------------------------------------
\tableofcontents
\newpage
%----------------------------------------------------------------------------------------------
\begin{comment}
HAUPTFORSCHUNGSGRAGE
Wie kann Künstlichen Intelligenz - zur Steigerung der Effizienz des Interior-Engineerings in der Unikatfertigung von VIP-Flugzeugkabinensystemen bei der Lufthansa Technik AG beitragen?

TEILFORSCHUNGSFRAGEN
2. Welche Automatisierungs-/Optimierungspotentiale können im Entwicklungsprozess von Interior Bauteilen mittels K.I. realisiert werden?
3. Welche K.I.-Technologie ist prädestiniert für die Ausschöpfung eines oder aller identifizierten Potentiale?
4. Wie könnte eine prototypische Entwicklung einer K.I. – Anwendung als Lösungskonzept in einer ver-
einfachten Prozessumgebung aussehen? 5. Welchen Effekt hat das K.I. gestützte Lösungskonzept auf die Effizienz im Entwicklungsprozess von
Interieur-Bauteilen?

$$$$$$$$______FORMALE ANFORDERUNGEN_________$$$$$$$
Master Thesis:          60 Seiten
Schriftart:             New Times Roman 12pt
                        Arail 11pt
                        Abbildungen & Tabellen 11pt
                        Fußnoten 10pt.

Zeilenabstand:          Text 1.5
                        Fußnote 1.0

Absatzformat:           Blocksatz

Nummerierung:           rechts Unten
                        Titelblatt keine Nummerierung
                        Text mit 1 beginnend (arab)
                        Verzeichnisse vor dem Text mit I (röm) - Inhalt, Abkürzung, Abb, Tabellen
                        Verzeichnisse nach dem Text fortl. (arab)
                        Verzeichnisse keine Überschriftnummer

$$$$$$$$______LITERATUR AUSWAHL_________$$$$$$$
Anzahl Quellen:         min: 1 pro Seite (unterschiedlich)
Sprache:                min 50% auf Englisch
Herkunft:               Wiss. Artikel in Journal min 30%
                        Internet max 10%
                        Monographien (Lehr- und Praktikerbücher) - rest

Literaturverz.:         Aplhabetisch - Chronologisch sortiert

$$$$$$$$______ABBILDUNGEN & TABELLEN_________$$$$$$$
Quellenangabe:          Bei jeder Tabelle und Abbildung
                        "Quelle: " muss führend davor stehen (Abbildung unten, Tabelle oben)
                        Eigene Darstellung: "Quelle: Eigenen Darstellung in Anlehung an"
\end{comment}
%----------------------------------------------------------------------------------------------
\section{Einführung}
\begin{comment}
- Thema und Aufgabenstellung
  ○ Thema
    § Aktuelle bzw. Historische Bezüge herstellen
    § Größeren Bezugsrahmen darstellen
      □ Vom allg. Meinen auf meine Aufgabenstellung kommen
  ○ Aufgabenstellung
    § Wichtigste Punkte aus der offiziellen Aufgabenstellung mit der eigenen Formulierung umschreiben (ggf. Offizielle AGST in die Arbeit einfügen)
    § Ziele der Untersuchung mit themenrelevanten Teilfragen
    § Abgrenzung des Untersuchungsgegenstandes mit Begründung
- Vorgehensweise
  ○ Gang der Untersuchung darlegen
    § Welche Schritte bin ich gegangen um an mein Ziel zu kommen
  ○ Vorgehensweise darlegen (Orientierung an der Gliederung)
  ○ Begründung zur Vorgehensweise liefern
    § Input warum ich so und nicht anders vorgegangen bin
- Ausschluss/ Eingrenzung
  ○ Beschreiben welche Themengebiete nich behandelt werden
    § Begründen
  ○ Relevanz des Themas unterstreichen
  ○ Nach Fertigstellung der Arbeit
    § Erneut prüfen
\end{comment}
Hallo Welt gäääähn \parencite[vgl.][S. 100]{Reich2013}
\newpage
%-------------------------------------------------------------------------------------------------------
\section{Stand der Technik}

\subsection{Beschreibung des Design Engineering Prozesses}

\subsection{Beschreibung des Design Engineering Prozesses der PD-VIP}

\subsection{Übersicht aktueller K.I. Technologien}

\subsection{Beschreibung bereits eingesetzter Technologien im Design Engineering Prozess der PD-VIP}

\newpage
%-------------------------------------------------------------------------------------------------------

\section{Hauptteil}

\subsection{Diskussion relevanter Wirkungspotentiale der differenzierten digitalen Technologien}

\subsection{Gerichtete Prozessanalyse (Fokus: Wirkungspotentiale der K.I.)}

\subsection{Evaluation und Entscheidung für eine K.I. – Technologie.}

\subsection{Entwicklung des Prozessmodells}

\subsection{Entwicklung des K.I.- gestützten Lösungskonzeptes}

\subsection{Evaluation des Lösungskonzeptes am Prozessmodell}

\subsection{Validierung anhand realer Completion-Projekte}

\newpage
%--------------------------------------------------------------------------------------------------------

\section{Schluss}

\subsection{Beantwortung der Forschungsfrage}

\subsection{Ausblick}

\newpage
%-------------------------------------------------------------------------------------------------------
\printbibliography
\end{document}
